\documentclass[a4paper, 12pt]{book}


\usepackage[a4paper, left=2.5cm, right=2.5cm, top=3cm, bottom=3cm]{geometry}
\usepackage{times}
\usepackage{afterpage}
\usepackage[utf8]{inputenc} %¡¡NO codificar en latin!!
\usepackage[spanish]{babel} % Comenta esta línea si tu memoria es en inglés
\usepackage{url}
%\usepackage[dvipdfm]{graphicx}
\usepackage{graphicx}
\usepackage{float}  %% H para posicionar figuras
\usepackage[nottoc, notlot, notlof, notindex]{tocbibind} %% Opciones de índice
\usepackage[none]{hyphenat}
\usepackage{cite}
\usepackage[pagestyles]{titlesec}
\usepackage{lipsum}
\usepackage[bookmarks = true, colorlinks=true, linkcolor = black, citecolor = black, menucolor = black, urlcolor = blue]{hyperref} %para que los enlaces no aparezcan recuadrados en rojo y cambia los colores de los enlaces

%para que el índice llegue hasta subsubsection
\setcounter{tocdepth}{3}
\setcounter{secnumdepth}{3}

\begin{document}

	\begin{titlepage}
		\centering
		\includegraphics[scale=0.5]{img/logo_urjc.jpg}
		\vspace{3cm}
		
		\Large
		GRADO EN INGENIERÍA EN ------ DE LAS TELECOMUNICACIONES
		
		\vspace{0.4cm}
		
		\large
		Curso Académico ----/----
		
		\vspace{0.8cm}
		
		Trabajo Fin de Grado
		
		\vspace{2.5cm}		
		\LARGE		
		TÍTULO DEL TRABAJO
		
		\vspace{2.5 cm}
		
		\large
		Autor :  \\
		Tutor : 
		\afterpage{\null\newpage}
		\pagestyle{empty}
	\end{titlepage}
	
	
\thispagestyle{empty}
\begin{flushright}
	\textit{Dedicado a \\
		mi familia, amigos y mascotas, que tando \\
		me han aguantado.\\
		Al café, la cerveza y el chocolate, que tanto\\
		me han ayudado.}
\end{flushright}
\afterpage{\null\newpage}
\pagestyle{empty}

	\chapter*{Agradecimientos}
\thispagestyle{empty}
\label{cap:agradecimientos}

Esto es un ejemplo. Ver en  \cite{darwin2009origen} \cite{del1984luces}

\afterpage{\null\newpage}
\pagestyle{empty}
	\chapter*{Resumen}
\thispagestyle{empty}
\label{cap:resumen}

Esto es un ejemplo. Ver en  \cite{darwin2009origen} \cite{del1984luces}

\afterpage{\null\newpage}
	
	\tableofcontents
	\pagenumbering{arabic}
	\setcounter{page}{1}
	\mainmatter

	\chapter{Introducción}
\label{cap:introduccion}

Esto es un ejemplo. Ver en  \cite{darwin2009origen} \cite{del1984luces}
	%cada vez que quieras un capitulo nuevo, 
	%descomentarlo y crear el archivo .tex	
%	\input{capitulo1}
%	\input{capitulo2}
%	\input{capitulo3}
%	\input{capitulo4}
%	\input{capitulo5}

	
	\bibliographystyle{acm} %para que se muestre la bibliografía
	\bibliography{bibliografia_tfg.bib}
	%En la bibliografía no sale nada que no se haya citado antes (se supone que si no lo citas es porque no lo has usado)
	
\end{document}